\documentclass[a4paper]{article}
\usepackage{enumitem}
\usepackage{hyperref}
\begin{document}
\title{GSoC'14: Improving PyBOMBS}
\author{Alick Zhao}
\date{\today}
\maketitle

\section{Contact Information}

\begin{itemize}[noitemsep]
  \item Name: Alick Zhao
  \item Chinese name: Tao ZHAO
  \item Email: \url{alick9188@gmail.com}
  \item IRC: alick at freenode
  \item Web page: \url{http://alick.me}
  \item Blog: \url{https://wp-awesome.rhcloud.com/}
  \item Location: Beijing, China (UTC+8)
\end{itemize}

\section{Background Information}

\subsection{Education}

I am a master student in the Department of Electronic Engineering, Tsinghua
University, Beijing, China. I major in wireless communications, and
software defined radio is among my research interests. I have published
one paper about software defined radio implementation upon
OpenBTS~\cite{zhao2013software}. I lead current lab research projects on
software defined radio, and maintain the
\href{http://network.ee.tsinghua.edu.cn/grwiki/}{project wiki}.

\subsection{Past experience with GNU Radio}

I have been using GNU Radio along with USRP, and
on the discuss-gnuradio mailing list since two years ago.
I used it to build a MIMO precoding system for my bachelor thesis.\footnote{%
The thesis is in Chinese and available at %
\url{http://cloud.alick.me/public.php?service=files&t=0a67a7f2d09a1b4385dc85018d487918}.}%
I submitted several bug reports and contributed one patch on the list.
I was not active for some time due to lab research focus shift, but I
always keep an eye on the GNU Radio updates.  I'm really excited about
the new features in recent versions of GNU Radio, and glad to see more and more
third-party open source modules about more interfacing devices, advanced
signal processing techniques, and modern wireless systems.

\subsection{Past experience with Other Open Source Projects}

Besides GNU Radio, I take part in quite a few open source projects.  I
am an active Fedora contributor. For now I regularly do translations,
chair IRC meetings, and organize offline events.  I would like to have
some packaging work, such as packaging GNU Radio related projects in the
near future. I am also an active member of TUNA, the open source league
in Tsinghua. I serve as a maintainer of the open source mirror site in
campus, a mentor for league newcomers, and an organizer and host of various
events. Recently I also work on \textsc{ThuThesis}, a \LaTeX\ thesis template
for Tsinghua University as well as my personal projects on
\href{https://github.com/alick9188}{Github}.

\section{Project Information}

PyBOMBS (Python Build Overlay Managed Bundle System) is the new install
management system for GNU Radio which can resolve the dependencies and
install out-of-tree GNU Radio projects. It provides a central, easy to
use tool to prepare the software defined radio development platform.
Although it is in early stage and not feature
complete, I think PyBOMBS has great potential to integrate the whole GNU
Radio ecosystem and lower the barrier to newcomers. That's why I want to
make it better through this GSoC project and future contributions.

\section{Suggested Improvements}

Below are a list of possible improvements suggested by the community
and my understanding to the points:

\begin{itemize}
  \item \emph{Develop documentation and tutorials:} Well-written
    documentation and tutorials about the usage of PyBOMBS system
    can make it easier for out-of-tree
    project developers to get along with PyBOMBS.
    We also need up to date documentations of API, coding style for
    PyBOMBS developers to contribute easily.
  \item \emph{More full featured GUI:} Currently the GUI just displays
    all available applications and all installed applications.
    More operations should be supported, and a search function is
    necessary concerning scalability. Besides the UI can be much
    improved.
  \item \emph{Improve app store:} To better present the out-of-tree
    projects, we need include metadata (author information, description,
    dependencies) and present them in the GUI. Automatic metadata
    extraction is appreciated. User rating will be a good plus.
  \item \emph{Integrate with CGRAN:} We have CGRAN\footnote{%
    \url{https://www.cgran.org/}} to host out-of-tree
    GNU Radio projects for quite some time. But its usage is not quite
    satisfactory. A nice integration between CGRAN and PyBOMBS can
    encourage usage of CGRAN, forming a better ecosystem. For example,
    required metadata field can be enforced when submitting projects to
    CGRAN. Besides, CGRAN might be the most suitable platform for user
    rating data as well as automatic building and testing.
  \item \emph{Test on additional operating systems and software
    configurations:} Testing is good. More tests will improve the code quality of
    PyBOMBS\@. It can also help to find bugs in out-of-tree projects.
    Currently PyBOMBS is mainly tested on Fedora, Ubuntu, and Debian,
    but it is necessary to ensure other distribution users can use it
    free of error.
  \item \emph{Develop automated test build systems:} We want to use
    automated tests to validate the working state of development
    branches of all packages, the compatibility of out-of-tree projects
    with new GNU Radio releases. The result
    could be used in app store, and exported to the CGRAN website.
  \item \emph{Core refinements:} For example, the distribution package
    handling code is better to be abstracted and unified to accommodate
    more distributions. We also need to handle the options of complex
    projects. The solution might be similar to the Gentoo's USE flag.
  \item \emph{Multiple version coexistence:} Currently different
    out-of-tree projects have different compatibility levels with GNU
    Radio. Since GNU Radio recently has quite a few API upgrades,
    some projects choose to use different develop branches to be
    compatible with different versions of GNU Radio. Therefore it is
    likely an end user will need to have multiple versions of GNU Radio
    installed, and use different versions with different out-of-tree
    projects. To address the issue, the implementation of
    virtualenv for Python and rvm for Ruby can hopefully give some guidance.
\end{itemize}

\section{Proposed Work for GSoC}

My proposed work items during GSoC period are as follows:

\begin{enumerate}
  \item \emph{Documentation and tutorials:} I want to begin with
    enriching the documentation for both PyBOMBS users and developers
    while I get familiar with it, which should be helpful to the
    project as well as my understanding of it. I will use Redmine Wiki
    for user documentation, and Sphinx for API documentation.
    Besides, a tutorial on the usage of PyBOMBS with a sample out-of\-
    tree project shall be created, which servers analogical purpose as
    GNU Radio's ``How to write a block''. The documentation will be
    updated along with the whole development phase.
  \item \emph{Abstract package handling code:} I will dive into the
    code handling binary packages in various Linux distributions
    (\nolinkurl{recipe.py}, \nolinkurl{sysutils.py}, etc.), and
    adopt the object oriented (OO) approach to abstract the code to
    expose a unified interface. In this way I can
    potentially add support for more distributions such as Arch and Gentoo
    with the help of package manager mapping information, for example
    the \href{https://wiki.archlinux.org/index.php/Pacman_Rosetta}%
    {Pacman Rosetta}.
  \item \emph{Improve app store:} I will come up with a list of
    necessary (or nice to have) metadata for each application and make a
    specification. Then I will investigate automatic ways of importing
    metadata from out-of-tree project upstream, for example using Github
    APIs or mechanisms on CGRAN to get various project information.
  \item \emph{Improve the GUI:} I will seek inspiration from GNOME
    Software\footnote{\url{https://live.gnome.org/Design/Apps/Software}}
    and various app stores, and come up with a modern design
    of the GUI. An interface of application details and the search
    functionality will be included. I will implement the design with
    PyQt4 graphic toolkit. I will focus on client search first (rather
    than querying remote server) and consider to use cache mechanism to
    imporve user experience.
  \item \emph{Packaging PyBOMBS for Linux distributions:} I will
    work with package maintainers of major Linux distributions to make
    PyBOMBS into their package repositories. The main concern is Fedora
    and Debian, but it is better to be included in Arch Linux and Gentoo
    as well.
\end{enumerate}

\section{Deliverables}

The deliverables of my GSoC project will be:
\begin{itemize}
  \item PyBOMBS documentation and tutorial.
  \item A unified API handling distribution packaging.
  \item App metadata specifications.
  \item A modern good-looking GUI for PyBOMBS.
  \item PyBOMBS packages in major Linux distributions.
\end{itemize}

\section{Timeline}

I will be available from April 21 to August 22 for GSoC project although
I have some other work or vacation.  I am an organizer of upcoming
FUDCon APAC 2014 on May 23--25, so my time for GSoC in starting period
might be less.  Later in summer I might have at most one week on
vacation away from coding and reliable network.  But overall I think I
can devote my summer time into the project and fulfil my tasks.  The
tentative agenda is illustrated in Table~\ref{tab:timeline}.

\begin{table}[htbp]
  \centering
  \begin{tabular}{lp{.7\textwidth}}
    \hline
    Time & Task \\
    \hline
    04/21 -- 04/27 & Initial discussion with the mentor \\
    04/28 -- 05/18 & Get familiar with current code, make simple fixes \\
    05/19 -- 05/25 & Create a tutorial about common PyBOMBS usage\\
    05/26 -- 06/01 & Enrich PyBOMBS documentation on wiki pages \\
    06/02 -- 06/08 & Setup Sphinx and generate API documentation \\
    06/09 -- 06/15 & Abstract API of distribution package handling code \\
    06/16 -- 06/22 & Create app metadata specification, investigate
    automatic metadata extraction ways\\
    06/23 -- 06/27 & Midterm evaluation submission\\
    06/28 -- 07/13 & Coding for metadata extraction and importation \\
    07/14 -- 07/20 & Design modern GUI, start packaging to distributions \\
    07/21 -- 08/10 & Coding for new GUI with PyQt4 \\
    08/11 -- 08/22 & Code clean-up, documentation updates, distribution
    packaging confirmation, and final evaluation submission \\
    \hline
  \end{tabular}
  \caption{Timeline of the GSoC project.}
  \label{tab:timeline}
\end{table}

\section*{Additional Information}

This is my first time to apply for a GSoC project, and I will appreciate
the opportunity to contribute to GNU Radio in this way.  The GSoC
project will do favor to my lab work, but it is not the lab's focus. It
will not be part of the lab work, and I won't get credit for it.
However, I will continue to contribute GNU Radio and other open source
projects after GSoC because it is my hobby and it benefits my study.

\begin{thebibliography}{1}

\bibitem{zhao2013software}
Tao Zhao, Pengkun Yang, Huimin Pan, Ruichen Deng, Sheng Zhou, and Zhisheng Niu.
\newblock Software defined radio implementation of signaling splitting in
  {Hyper-Cellular} network.
\newblock In {\em Proceedings of the Second Workshop of Software Radio
  Implementation Forum (SRIF 2013)}, SRIF '13, pages 81--84, New York, NY, USA,
  August 2013. ACM.
\newblock \href {http://dx.doi.org/10.1145/2491246.2491258}
  {\path{doi:10.1145/2491246.2491258}}.

\end{thebibliography}

\end{document}
% vim:set sw=2 tw=72:
