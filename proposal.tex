\documentclass[a4paper]{article}
\begin{document}
\title{GSoC'14: Improving PyBOMBS}
\author{Alick Zhao}
\date{\today}
\maketitle

\section{Introduction}

PyBOMBS (Python Build Overlay Managed Bundle System) is the new install
management system for GNU Radio which can resolve the dependencies and
install out-of-tree GNU Radio projects.

\section{Suggested Improvements}

Below are a list of possible improvements suggested by the mentor Tim O'Shea:

\begin{itemize}
  \item \emph{Develop documentation and tutorials:} Well-written
    documentation and tutorials about the usage of PyBOMBS system
    can make it easier for out-of-tree
    project developers to get along with PyBOMBS.
    We also need up to date documentations of API, coding style for
    PyBOMBS developers to contribute easily.
  \item \emph{More full featured GUI:} Currently the GUI just displays
    all available applications and all installed applications.
    More operations should be supported, and a search function is
    necessary concerning scalability. Besides the UI can be much
    improved.
  \item \emph{Improve app store:} To better present the out-of-tree
    projects, we need include metadata (author information, description,
    dependencies) and present them in the GUI. Automatic metadata
    extraction is appreciated. User rating will be a good plus.
  \item \emph{Integrate with CGRAN:} We have CGRAN to host out-of-tree
    GNU Radio projects for quite some time. But its usage is not quite
    satisfactory.
  \item \emph{Test on additional operating systems and software
    configurations:} Testing is good. More tests will improve the code quality of
    PyBOMBS\@. It can also help to find bugs in out-of-tree projects.
  \item \emph{Develop automated test build systems:} We want to use
    automated tests to validate the working state of development
    branches of all packages, the compatibility of out-of-tree projects
    with new GNU Radio releases. The result
    could be used in app store, and exported to the CGRAN website.
\end{itemize}

\section{Proposed Work for GSoC}

My proposed work items during GSoC period are as follows:

\begin{enumerate}
  \item \emph{Documentation and tutorials:} I want to begin with
    enriching the documentation for both PyBOMBS users and developers
    while I get familiar with it, which should be helpful to the
    project as well as my understanding of it. I will use Redmine Wiki
    for user documentation, and Sphinx for API documentation.
    Besides, a tutorial on the usage of PyBOMBS with a sample out-of\-
    tree project shall be created, which servers analogical purpose as
    GNU Radio's ``How to write a block''. The documentation will be
    updated along with the development cycle.
  \item \emph{Improve app store:} I will come up with a list of
    necessary (or nice to have) metadata for each application and make a
    specification. Then I will investigate automatic ways of importing
    metadata from out-of-tree project upstream, for example using Github
    APIs to get various repository information for projects hosted on
    Github.
  \item \emph{Improve the GUI:} I will seek inspiration from GNOME
    Applicatons and various app stores, and come up with a modern design
    of the GUI. An interface of application details and the search
    functionality will be included. I will implement the design with
    PyQt4.
\end{enumerate}

\section{Timeline}
\section{Conclusion}

\end{document}
% vim:set sw=2 tw=72:
