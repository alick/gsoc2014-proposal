\documentclass[a4paper]{article}
\usepackage{hyperref}
\begin{document}
\title{GSoC'14: Improving PyBOMBS}
\author{Alick Zhao}
\date{\today}
\maketitle

\section{Contact Information}

\begin{itemize}
  \item Name: Alick Zhao
  \item Email: \url{alick9188@gmail.com}
  \item IRC: alick at freenode
  \item Web Page: \url{http://alick.me}
  \item Location: Beijing, China (UTC+8)
\end{itemize}

\section{Background Information}

I am a master student in Department of Electronic Engineering, Tsinghua
University, Beijing, China. I major in wireless communications, and
software defined radio is among my research interests. I have published
one paper about software defined radio implementation upon
OpenBTS~\cite{zhao2013software}.

\subsection{Past experience with GNU Radio}

I have been using GNU Radio along with USRP, and
on the discuss-gnuradio mailing list since two years ago.
I used it to build a MIMO precoding system for my bachelor thesis.\footnote{%
The thesis is in Chinese and available at %
\url{http://cloud.alick.me/public.php?service=files&t=0a67a7f2d09a1b4385dc85018d487918}.}%
I submitted several bug reports and contributed one patch on the list.
I was not active for some time due to lab research focus shift, but I
always keep an eye on the GNU Radio updates.  I'm really excited about
the new features in recent versions of GNU Radio, and glad to see more and more
third-party open source modules about more interfacing devices, advanced
signal processing techniques, and wireless systems.

\subsection{Past experience with Open Source Projects}

Besides GNU Radio, I take part in quite a few open source projects.  I
am an active Fedora contributor. For now I regularly do translations,
chair IRC meetings, and organize offline events.  I would like to have
some packaging work, such as packaging GNU Radio related projects in the
near future. I am also an active member of TUNA, the open source league
in Tsinghua. I serve as a maintainer of the open source mirror site in
campus, a mentor for newcomers, and an organizer and host of various
events. Recently I also work on \textsc{ThuThesis}, a \LaTeX\ thesis template
for Tsinghua University as well as my personal projects on
\href{https://github.com/alick9188}{Github}.

\section{Project Information}

PyBOMBS (Python Build Overlay Managed Bundle System) is the new install
management system for GNU Radio which can resolve the dependencies and
install out-of-tree GNU Radio projects. It provides a central, easy to
use tool to prepare the software defined radio development platform.
Although the development is in early stage and it is not feature
complete, I think PyBOMBS has great potential to integrate the whole GNU
Radio ecosystem and lower the barrier to newcomers. That's why I want to
make it better through this GSoC project and afterwards.

\section{Suggested Improvements}

Below are a list of possible improvements suggested by the mentor Tim O'Shea:

\begin{itemize}
  \item \emph{Develop documentation and tutorials:} Well-written
    documentation and tutorials about the usage of PyBOMBS system
    can make it easier for out-of-tree
    project developers to get along with PyBOMBS.
    We also need up to date documentations of API, coding style for
    PyBOMBS developers to contribute easily.
  \item \emph{More full featured GUI:} Currently the GUI just displays
    all available applications and all installed applications.
    More operations should be supported, and a search function is
    necessary concerning scalability. Besides the UI can be much
    improved.
  \item \emph{Improve app store:} To better present the out-of-tree
    projects, we need include metadata (author information, description,
    dependencies) and present them in the GUI. Automatic metadata
    extraction is appreciated. User rating will be a good plus.
  \item \emph{Integrate with CGRAN:} We have CGRAN\footnote{%
    \url{https://www.cgran.org/}} to host out-of-tree
    GNU Radio projects for quite some time. But its usage is not quite
    satisfactory. A nice integration between CGRAN and PyBOMBS can
    encourage usage of CGRAN, forming a better ecosystem.
  \item \emph{Test on additional operating systems and software
    configurations:} Testing is good. More tests will improve the code quality of
    PyBOMBS\@. It can also help to find bugs in out-of-tree projects.
  \item \emph{Develop automated test build systems:} We want to use
    automated tests to validate the working state of development
    branches of all packages, the compatibility of out-of-tree projects
    with new GNU Radio releases. The result
    could be used in app store, and exported to the CGRAN website.
\end{itemize}

\section{Proposed Work for GSoC}

My proposed work items during GSoC period are as follows:

\begin{enumerate}
  \item \emph{Documentation and tutorials:} I want to begin with
    enriching the documentation for both PyBOMBS users and developers
    while I get familiar with it, which should be helpful to the
    project as well as my understanding of it. I will use Redmine Wiki
    for user documentation, and Sphinx for API documentation.
    Besides, a tutorial on the usage of PyBOMBS with a sample out-of\-
    tree project shall be created, which servers analogical purpose as
    GNU Radio's ``How to write a block''. The documentation will be
    updated along with the development cycle.
  \item \emph{Improve app store:} I will come up with a list of
    necessary (or nice to have) metadata for each application and make a
    specification. Then I will investigate automatic ways of importing
    metadata from out-of-tree project upstream, for example using Github
    APIs to get various repository information for projects hosted on
    Github.
  \item \emph{Improve the GUI:} I will seek inspiration from GNOME
    Software\footnote{https://live.gnome.org/Design/Apps/Software}
    and various app stores, and come up with a modern design
    of the GUI. An interface of application details and the search
    functionality will be included. I will implement the design with
    PyQt4.
\end{enumerate}

\section{Deliverables}

The deliverables of my GSoC project will be:
\begin{itemize}
  \item PyBOMBS documentation and tutorial.
  \item App metadata specifications.
  \item A modern good-looking GUI for PyBOMBS.
\end{itemize}

\section{Timeline}

I will be available from April 21 to August 22 to ensure time commitment
for GSoC project.  I am an organizer of upcoming FUDCon APAC 2014 on May
23--25, so my time for GSoC in that period might be less.  Later I might
have at most one week on vacation away from coding and reliable network.
But overall I think I can devote my summer time into the project.
The tentative agenda is illustrated in Table~\ref{tab:timeline}.

\begin{table}[htbp]
  \centering
  \begin{tabular}{lp{.7\textwidth}}
    \hline
    Time & Task \\
    \hline
    04/21 -- 04/27 & Initial discussion with the mentor \\
    04/28 -- 05/18 & Get familiar with current code, make simple fixes \\
    05/19 -- 05/25 & Create a tutorial about common PyBOMBS usage\\
    05/26 -- 06/01 & Enrich PyBOMBS documentation on wiki pages \\
    06/02 -- 06/08 & Setup Sphinx and generate API documentation \\
    06/09 -- 06/22 & Create metadata specification, investigate
    automatic metadata extraction ways\\
    06/23 -- 06/27 & Midterm evaluation submission\\
    06/28 -- 07/13 & Coding for metadata extraction and importation \\
    07/14 -- 07/20 & Design new GUI, get started with PyQt4 \\
    07/21 -- 08/10 & Coding for new GUI with PyQt4 \\
    08/11 -- 08/22 & Code clean-up, and final evaluation submission \\
    \hline
  \end{tabular}
  \caption{Timeline of the GSoC project.}
  \label{tab:timeline}
\end{table}

\begin{thebibliography}{1}

\bibitem{zhao2013software}
Tao Zhao, Pengkun Yang, Huimin Pan, Ruichen Deng, Sheng Zhou, and Zhisheng Niu.
\newblock Software defined radio implementation of signaling splitting in
  {Hyper-Cellular} network.
\newblock In {\em Proceedings of the Second Workshop of Software Radio
  Implementation Forum (SRIF 2013)}, SRIF '13, pages 81--84, New York, NY, USA,
  August 2013. ACM.
\newblock \href {http://dx.doi.org/10.1145/2491246.2491258}
  {\path{doi:10.1145/2491246.2491258}}.

\end{thebibliography}

\end{document}
% vim:set sw=2 tw=72:
